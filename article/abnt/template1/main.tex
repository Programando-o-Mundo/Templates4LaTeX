\documentclass[
  % -- opções da classe memoir --
  12pt,				         % tamanho da fonte
  oneside,			       % para impressão apenas no recto. Oposto a twoside
  a4paper,			       % tamanho do papel. 
  article,
  % -- opções da classe abntex2 --
  %chapter=TITLE,		   % títulos de capítulos convertidos em letras maiúsculas
  %section=TITLE,		   % títulos de seções convertidos em letras maiúsculas
  %subsection=TITLE,	 % títulos de subseções convertidos em letras maiúsculas
  %subsubsection=TITLE % títulos de subsubseções convertidos em letras maiúsculas
  % -- opções do pacote babel --
  english,		       	 % idioma adicional para hifenização
  brazil,			      	 % o último idioma é o principal do documento
]{abntex2}

\usepackage{mypackage}

\titulo{Article title}
\autor{Author name \and Author name \and Author name \and Author name}
\local{Local Name}
\data{YEAR}
\instituicao{%
  Institution
  }
%\tipotrabalho{NOT NECESSARY}
% ---
% 
% ---
% informações do PDF
\makeatletter
\hypersetup{
     	%pagebackref=true,
		pdftitle={\@title}, 
		pdfauthor={\@author},
    	pdfsubject={Insert title here},
	    pdfcreator={authors},
		pdfkeywords={keyword}{keyword1}{keyword2}{keyword3}{keyword4}, 
		colorlinks=true,       		% false: boxed links; true: colored links
    	linkcolor=black,          	% color of internal links
    	citecolor=blue,        		% color of links to bibliography
    	filecolor=magenta,      		% color of file links
		urlcolor=blue,
		bookmarksdepth=4
}
\makeatother

\makeindex

% ---
% Iniciando efetivamente o documento
% ---
\begin{document} 

    % Fazer com que as secções sejão subcapitulos
    \renewcommand{\thesection}{\noindent\arabic{chapter}.\arabic{section}}
    % ---

    % Selecionando linguagem
    % ---

    \selectlanguage{brazil}
    % ---
    % Retira espaço extra obsoleto entre as frases.
    % ---

    \frenchspacing
    % ---
    % Imprimir a capa 
    % ---
    
    \imprimircapa
    % ---
    % Imprimir a tabela de conteúdos(Sumário)
    % ---
    \pdfbookmark[0]{\contentsname}{toc}
    \tableofcontents*
    \cleardoublepage
    % ---
    % PARTE TEXTUAL
    % ---
    \textual
    % ---
    % Criar nova página e então iniciar a escrita
    % ---
    \newpage
    
    %--------------Introduction----------------%

    \import{./sections/introduction}{introduction.tex}

    %--------------Development-----------------%

    \import{./sections/development}{1.tex}
    
    \import{./sections/development}{2.tex}

    \import{./sections/development}{3.tex}

    \import{./sections/development}{4.tex}

    \import{./sections/development}{5.tex}

    %--------------Conclusion------------------%

    \import{./sections/conclusion}{conclusion.tex}

    \postextual

    %--------------References------------------%

    \bibliography{references}

    %--------------Appendex--------------------%

    \import{./sections/appendix}{appendix.tex}

    \phantompart

    \printindex

\end{document}



\section{Metodologia ou Proposta de Arquiteturas}

    %A metodologia utilizada no artigo consiste no estudo de artigos que se correlacionam com o tema estudado, 
    %como também a utilização e desenvolvimento de arquiteturas feitas no simulador Amnesia.
    Para a elaboração de teste utilizamos o simulador-Amnésia para elaborar algumas
    arquiteturas (tabelas \ref{tab1} e \ref{tab2}) que a partir desses cenários serem executados
    executamos os testes (tabela \ref{tab3})
    \begin{table}[H]
    \caption{características gerais da arquitetura i}
        \centering
        \begin{tabular}{|c|c|c|c|}
            \hline
            \textbf{Especificações} & \multicolumn{3}{|c|}{\textbf{Partes da Arquitetura}} \\
            \cline{2-4} 
            \textbf{da Arquitetura} & \textbf{\textit{Processador}}& \textbf{\textit{CPU}}& \textbf{\textit{Trace}} \\
            \hline
            Tamanho da palavra & --- & 4 & 4 \\
            \hline
            processorContains & 0 & --- & --- \\
            \hline
            Ciclos por escrita & 0 & --- & --- \\
            \hline
        \end{tabular}
        \label{tab1}
    \end{table}

    \begin{table}[H]
    \caption{características gerais da arquitetura ii}
    \centering
        \begin{tabular}{|c|c|c|}
            \hline
            \textbf{Especificações} & \multicolumn{2}{|c|}{\textbf{Partes da Arquitetura}} \\
            \cline{2-3} 
            \textbf{da Arquitetura} & \textbf{\textit{Memória Principal}}& \textbf{\textit{Cache}} \\
            \hline
            Tamanho da linha / bloco & 1 & 1  \\
            \hline
            Ciclos por leitura & 1 & 1  \\
            \hline
            Ciclos por escrita & 2 & 2  \\
            \hline
            Tempo do ciclo & 10 & 1  \\
            \hline
            Tamanho da memória & 16 & {$^{\mathrm{*}}$}2  \\
            \hline
            Associatividade & --- & 2  \\
            \hline
            Politica de escrita & --- & Write-Through  \\
            \hline
            Politica de substituição & --- & {$^{\mathrm{*}}$}FIFO  \\
            \hline
            \multicolumn{3}{l}{$^{\mathrm{*}}$ Os valores com o asterisco foram os valores modificados.}
        \end{tabular}
        \label{tab2}
    \end{table}

    \begin{table}[H]
    \caption{Dados modificados}
    \centering
        \begin{tabular}{|c|c|c|}
            \hline
            \textbf{Especificações} & \multicolumn{2}{|c|}{\textbf{Valores}} \\
            \cline{2-3} 
            \textbf{da Arquitetura} & \textbf{Tamanho da Memoria} & \textbf{Politica de Substituição} \\
            \hline
            Cenário 1 & 2 & FIFO \\
            \hline
            Cenário 2 & 2 & LRU \\
            \hline
            Cenário 3 & 4 & FIFO\\
            \hline
            Cenário 4 & 4 & LRU\\
            \hline
            Cenário 5 & 8 & FIFO\\
            \hline
            Cenário 6 & 8 & LRU\\
            \hline
            Cenário 7 & 16 & FIFO\\
            \hline
            Cenário 8 & 16 & LRU\\
            \hline
            %\multicolumn{3}{l} {Sample of a Table footnote.}
        \end{tabular}
        \label{tab3}
    \end{table}

    Ao modificar o valor do tamanho da memória e da política de substituição da cache,
    é possível criar 8 cenários distintos para se avaliar o impacto do tamanho da cache
    no desempenho dos métodos de substituição. Sendo assim, com os cenários propostos
    foram realizados diversos testes com o intuito de analisar a localidade temporal e
    localidade espacial nas arquiteturas.
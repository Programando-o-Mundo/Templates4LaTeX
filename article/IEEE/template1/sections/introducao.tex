\section{Introdução}

    %Espera-se que nesta seção, seja escrita uma contextualização inicial 
    %para situar o leitor. Ou seja, saber em qual área ele está "pisando".
    %No contexto discutido, é apresentado um problema em aberto (um problema 
    %pode ser escrito como uma pergunta), em seguida um objetivo para "responder" 
    %ao problema descrito. Depois, a contribuição científica é descrita, e por fim, 
    %um parágrafo para descrever a organização do artigo é feito.

    A Cache é um dispositivo de acesso rápido, que fica localizado dentro do processador,
    com a intenção de reduzir o acesso do processador à memória principal, que demanda um
    tempo de acesso muito superior à cache. Criando uma referência da localidade do dado na 
    memória principal, até mesmo gravando esses dados, porém existe um limite de quantos 
    dados podem ser armazenados. Para isso é necessário uma política de armazenamento para 
    reocupar espaço quando necessário.

    Desde a invenção dos computadores, a busca por otimizar os processos que são executados 
    nos hardwares se tornou foco, considerando que simples ajustes podem render grandes 
    ganhos de desempenho. Dessa forma, um algoritmo eficiente na leitura e escrita de 
    dados na cache é de grande importância. 

    Com isso em mente, pensamos em políticas de substituição clássicos, LRU e o FIFO, para
    averiguar qual possui melhor desempenho em diferentes cenários. Começamos pensando em 
    aumentar gradativamente o número de instruções que acessam as palavras na RAM para serem 
    escritas na cache. Dessa forma, seria simples visualizar como o número de instruções
    influencia no tempo de acesso aos dados. Além disso, pensamos também em vários cenários 
    com caches de diferentes tamanhos, desde o mínimo de espaço possível, até uma cache 
    de mesmo tamanho que a memória RAM.

    O resto do papel está organizado da seguinte maneira. Seção 2 irá mostrar pesquisas correlatas,
    apresentar de maneira breve o assunto de cada e se existe algo que possa contribuir com a pesquisa atual.
    Seção 3 apresenta as arquiteturas utilizadas para a realização dos testes, assim também com os cenários 
    que foram testados. Resultados experimentais são colocados na Seção 4. Por fim, a Seção 5 dá conclusão 
    a nossa pesquisa.